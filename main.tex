\documentclass[unicode,12pt,aspectratio=169,dvipdfmx]{beamer}
\usepackage{bxdpx-beamer}
\usetheme[progressbar=frametitle]{metropolis}
\renewcommand{\kanjifamilydefault}{\gtdefault}
\usepackage{bm}
\title{\textbf{介護における音響HARと連合学習を用いた異常検知}}
\author{\textbf{竹本志恩}}
\institute{\textbf{INIAD}}
\date{\today}
% セクション開始時にミニTOC
% \AtBeginSection[]{
%   \begin{frame}[plain]
%     \frametitle{目次}
%     \tableofcontents[currentsection, hideallsubsections]
%   \end{frame}
% }
\begin{document}
%----------------------------------------
% 本編スライド
%----------------------------------------
% Slide 1: Title
\begin{frame}
  \titlepage
\end{frame}
% Slide 2: Agenda
% \begin{frame}{アジェンダ}
%   \begin{enumerate}
%     \item 背景・研究目的
%     \item 本研究の独自性
%     \item 音響HARの基礎と課題
%     \item FLの基礎とHARへの適用
%     \item FL-HARの主要課題
%     \item 課題解決の戦略
%     \item 関連研究動向
%     \item 今後の展望
%     \item まとめ・Q\&A
%   \end{enumerate}
% \end{frame}
% Slide 3: Background & Objectives
% \section{背景・研究目的}
\begin{frame}{背景と研究目的}
  \begin{itemize}
    \item 高齢化社会における介護現場の負担増大
    \item HARに着目: 介護者の負担軽減
    \item センサー/音響データのプライバシー懸念
    \item 目的:異常イベントをリアルタイム検知+通知
  \end{itemize}
\end{frame}
% Slide 4: Uniqueness
% \section{本研究の独自性}
\begin{frame}{本研究の独自性}
  \begin{itemize}
    \item 複数の異常イベントを同時/複合的に検知
    \item マルチラベル分類で音響イベントを解釈
    
    $|$-例1: 咳と苦痛の声が同時に聞こえる$>$重篤な状態
    
    $|$-例2: 転倒に似た音を,他のイベントを基に判断
    \item 深い状況把握と迅速対応支援を実現
  \end{itemize}
\end{frame}
% Slide 5: Acoustic HAR Basics
% \section{音響HARの基礎と課題}
\begin{frame}{音響HARの基礎と課題}
  \begin{itemize}
    \item 非接触モニタリング+視覚プライバシー保護
    \item 環境騒音によるマスキング
    \item 高静音性活動の検知が困難
    \item 音の多様性/同時発生の問題
    
    $>$ 頑健なモデルが必要
  \end{itemize}
\end{frame}
% Slide 6: FL Basics & Benefits
% \section{FLの基礎とHAR適用}
\begin{frame}{連合学習 (FL) の基礎とHARへの適用}
  \begin{itemize}
    \item 分散学習:生データ非共有でモデル協調学習
    \item プライバシー保護、パーソナル化、通信効率
    \item 介護現場に適した技術
  \end{itemize}
\end{frame}
% Slide 7: Challenges I (Data)
\section{FL-HARの主要課題}
\begin{frame}{主要課題①:データ関連}
  \begin{itemize}
    \item 極端なデータ不均衡(稀イベント、多ラベル共起)

    $|-$複数異常の組み合わせはほぼない
    \item Non-IID分布
    
    $|-$各利用者の音響特性や行動の差
    \item ラベルの関係を捉えたモデリング
    \item マルチラベル異常データセットの不足
  \end{itemize}
\end{frame}
% Slide 8: Challenges II (Model, Comm, Privacy)
\begin{frame}{主要課題②:モデル・通信・プライバシー}
  \begin{itemize}
    \item モデル複雑化 $<$$-$$>$ 通信オーバーヘッド
    \item リアルタイム通知の遅延リスク
    \item 稀な複合イベントの機密性への影響
  \end{itemize}
\end{frame}
% Slide 9: Challenges III (Personalization & CL)
\begin{frame}{主要課題③:パーソナライズ化・継続学習}
  \begin{itemize}
    \item 個人差対応のパーソナライズドFL
    \item コールドスタート:未知の音響組み合わせ
    \item 音響環境変化への適応
  \end{itemize}
\end{frame}
% Slide 10: Strategy I (Data & Non-IID)
\section{課題解決の戦略}
\begin{frame}{対策①:データ不均衡・Non-IID対策}
  \begin{itemize}
    \item 合成データ生成 (GAN, SMOTE)
    \item 少数クラスを強調: Time-Balanced Focal Loss など特殊損失
    \item パーソナライズドFL: FedProx, SCAFFOLD
    \item モデル統合の効率化: クラスタリングFL/ドメイン適応
  \end{itemize}
\end{frame}
% Slide 11: Strategy II (Privacy)
\begin{frame}{対策②:プライバシー強化}
  \begin{itemize}
    \item オンデバイス特徴抽出によるデータ最小化
    \item ノイズ付与: 差分プライバシー (DP)
    \item 暗号化集約: セキュアアグリゲーション (SA)/同型暗号 (HE)
    \item 精度とオーバーヘッドのバランス
  \end{itemize}
\end{frame}
% Slide 12: Strategy III (Model & Efficiency)
\begin{frame}{対策③:モデル・リソース効率化}
  \begin{itemize}
    \item モデル軽量化: モデル圧縮/量子化/枝刈り
    \item 計算の分割: Federated Split Learning (FSL)
    \item 通信量削減: 勾配圧縮・選択的更新
    % \item マルチモーダル融合戦略
  \end{itemize}
\end{frame}
% Slide 13: Related Work
% \section{関連研究動向}
% \begin{frame}{関連研究:マルチラベルFL}
%   \begin{itemize}
%     \item FMLL, FedMLP, FedLGT の概略
%     \item HAR異常検知への適用可能性
%     \item 課題:HAR特有の極端不均衡、時間相関
%   \end{itemize}
% \end{frame}
% Slide 14: Future Directions
% \section{今後の展望}
\begin{frame}{今後の研究展望}
  \begin{itemize}
    \item 標準化データセットの整備
    \item FL対応マルチラベル損失/新アーキテクチャ開発

    $|-$マルチラベルFLなど参考
    \item 連合継続学習
    \item 因果推論の検討
  \end{itemize}
\end{frame}
% Slide 15: Conclusion & Q&A
% \section{まとめ・結論}
% \begin{frame}{まとめ}
%   \begin{itemize}
%     \item マルチラベル異常検知FL-HARの可能性
%     \item 主要課題と解決の方向性
%     \item 社会実装へのロードマップ
%   \end{itemize}
%   \vfill
% \end{frame}
%----------------------------------------
% 補足資料スライド (配布用)
%----------------------------------------
\appendix 

\section{補足資料}
% \subsection{データ不均衡・Non-IIDへの対応}
\begin{frame}{データ不均衡・Non-IIDへの対応:SMOTE}
\begin{itemize}
    \item \textbf{概要}:\textbf{Synthetic Minority Over-sampling Technique}の略
    \item \textbf{目的}:\textbf{少数クラスのデータ不足}に起因する不均衡を緩和
    \item \textbf{動作原理}:既存の少数サンプル間の特徴空間で補間することで、\textbf{合成サンプルを生成}。音響データの場合、MFCCなどの\textbf{抽出された音響特徴量}に適用
    \item \textbf{利点と課題}:
    \begin{itemize}
        \item 実装が比較的容易で、多くのライブラリで利用可能
        \item 単純な適用では\textbf{過学習}の可能性
        \item 音響的に意味のある合成サンプルを生成できない、時間的連続性が破壊される可能性も
        \item \textbf{希少イベント固有の音響特性}を保持することが重要
    \end{itemize}
\end{itemize}
\end{frame}

\begin{frame}{データ不均衡・Non-IIDへの対応:Time-Balanced Focal Loss}
\begin{itemize}
    \item \textbf{概要}:持続時間のばらつきによるデータ不均衡を考えた損失関数
    \item \textbf{目的}:イベントの発生頻度だけでなく、\textbf{持続時間も考慮}
    \item \textbf{動作原理}:
    \begin{itemize}
        \item 標準的なFocal Lossをクラスごとの重み $w_c$で拡張
        \item $w_c$は、データ量(音源/クリップ数)とイベント持続時間(フレームの比率$r_c$)を考慮
        \item \textbf{希少で持続時間の短い音響イベントの学習が強化され、検出精度が向上}
        % \item 式は以下
        %     \begin{equation*}
        %     \text{TBFL}(p,y) = -\sum_{c} w_c \left\{ y_c (1-p_c)^\gamma \log(p_c) + (1-y_c) p_c^\gamma \log(1-p_c) \right\}
        %     \end{equation*}
        %     ここで、$w_c \propto \frac{1-\beta^{c k} \times r_c}{1-\beta^{c}}$であり、$\sum_c w_c = C$($C$はターゲットクラス数)です。
    \end{itemize}
\end{itemize}
\end{frame}

% \begin{frame}{データ不均衡・Non-IIDへの対応:ドメイン適応}
% \begin{itemize}
%     \item \textbf{概要}:\textbf{異なるデータ分布(ドメイン)間}でのモデルの汎化能力を高めるための技術です [16, 23]。
%     \item \textbf{目的}:Federated Learningにおいて、\textbf{クライアントごとのデータ分布が異なるNon-IID問題}に対処し、モデルの適応性を向上させます [16, 24]。
%     \item \textbf{動作原理}:
%     \begin{itemize}
%         \item データセット内の\textbf{位相的構造(topological structures)を整合させる}ことにより、限られたラベル付きデータでの性能を向上させます [24]。
%         \item \textbf{クラスタリングFL}(類似したデータ分布や異常パターンを持つクライアントをグループ化し、グループごとにモデルを学習させるアプローチ)もNon-IIDデータへの対応戦略として挙げられています [16, 17, 25]。
%         \item ClusterFLは、\textbf{類似性認識型連合学習システム}としてHARの異質性に対処します [25]。
%     \end{itemize}
% \end{itemize}
% \end{frame}

% \subsection{プライバシー強化技術}
\begin{frame}{プライバシー強化技術:差分プライバシー (DP)}
\begin{itemize}
    \item \textbf{目的}:個々のデータサンプルの寄与を曖昧にし、漏洩リスクを保証
    \item \textbf{動作原理}:
    \begin{itemize}
        \item モデル更新(勾配や重みなど)に\textbf{ノイズを付加}
        \item Federated Split Learning (FSL) では、中間活性化情報に\textbf{ガウスノイズ}
    \end{itemize}
    \item \textbf{課題}:以下のトレードオフ
    \begin{itemize}
        \item モデルの精度低下
        \item 計算・通信オーバーヘッドの増加
    \end{itemize}
\end{itemize}
\end{frame}

% \begin{frame}{プライバシー強化技術:セキュアアグリゲーション (SA)}
% \begin{itemize}
%     \item \textbf{概要}:Federated Learningにおける\textbf{プライバシー強化技術(PETs)}の一つです [16, 23, 26]。
%     \item \textbf{目的}:各クライアントからのモデル更新を、\textbf{個々の更新内容をサーバーに知られることなく集約}することを目指します [16, 26]。
%     \item \textbf{動作原理}:複数のクライアントからのモデルパラメータを安全に結合し、サーバーは最終的な集約結果のみを得る仕組みを提供します [26]。
%     \item \textbf{利点と課題}:
%     \begin{itemize}
%         \item サーバーによる\textbf{個別のクライアントデータ推論のリスクを低減}し、プライバシー保護に貢献します [26]。
%         \item しかし、\textbf{学習時間が長くなる}可能性があると指摘されています [27]。
%     \end{itemize}
% \end{itemize}
% \end{frame}

% \begin{frame}{プライバシー強化技術:準同型暗号 (HE)}
% \begin{itemize}
%     \item \textbf{目的}:\textbf{暗号化されたデータに対して直接計算を実行}することを可能にし、元のデータを復号することなくモデルを訓練
%     \item \textbf{動作原理}:クライアントは自身のデータを暗号化してモデル更新を送信し、サーバーは暗号化された更新をそのまま集約し、その結果も暗号化されたままクライアントに返す。復号はクライアント側で実施
%     \item \textbf{利点と課題}:
%     \begin{itemize}
%         \item \textbf{非常に強固なプライバシー保護}を提供します [26]。
%         \item しかし、\textbf{計算コストや通信オーバーヘッドが非常に高い}傾向があり [26]、モデルの\textbf{精度低下とのトレードオフ}も考慮する必要があります [26]。
%     \end{itemize}
% \end{itemize}
% \end{frame}

% \subsection{リソース効率化技術}
\begin{frame}{リソース効率化技術:Federated Split Learning (FSL)}
\begin{itemize}
    \item \textbf{概要}:\textbf{モデルの計算負荷をクライアントとサーバーで分割}
    \item \textbf{目的}:\textbf{エッジデバイスの計算能力、メモリ、通信帯域の制約を両立}させながら、モデル訓練や推論を効率的に実行
    \item \textbf{動作原理}:
    \begin{itemize}
        \item モデルの層を分割点(\textbf{`cut layer`}または\textbf{`split layer`})で分け、\textbf{クライアント側でモデルの一部(例:特徴抽出層)を処理}し、\textbf{中間活性化情報}をサーバーに送信
        \item \textbf{サーバー側で残りのモデル処理(例:分類層)、損失計算、およびバックプロパゲーションを実行}
        \item 例: クライアント側にLSTM層、サーバー側に出力層(Dense+Softmax)を配置
    \end{itemize}
\end{itemize}
\end{frame}

% \subsection{マルチラベルFLフレームワークの概要とHAR適用課題}
\begin{frame}{マルチラベルFLフレームワーク:FMLL}
\begin{itemize}
    \item \textbf{動作原理}:
    \begin{itemize}
        \item マルチラベルデータセットを\textbf{バイナリ関連性 (Binary Relevance) 戦略}に基づき複数の二値データセットに分解
        \item 各二値データセットで\textbf{ローカルモデル(例: REPTree)を訓練}し、中央ノードでこれらのローカルモデルを\textbf{集約}してグローバルモデルを作成
    \end{itemize}
    \item \textbf{HAR適用時の課題}:
    \begin{itemize}
        \item 現状、\textbf{動物科学分野のデータセット}で検証されており [33, 35]、\textbf{HAR異常データへの直接的な適用性は未検証}
        \item ベース分類器としてREPTreeを使用しており,表現力の限界が懸念
    \end{itemize}
\end{itemize}
\end{frame}

\begin{frame}{マルチラベルFLフレームワーク:FedMLP}
\begin{itemize}
    \item \textbf{概要}:{部分的なアノテーション(欠損ラベル)を持つマルチラベルFL}に対応
    % \item \textbf{動作原理}:
    % \begin{itemize}
    %     \item \textbf{第1段階}:ウォームアップモデルで\textbf{クラスプロトタイプを生成}し、自己適応型閾値 (ST) による\textbf{疑似ラベリングを用いて欠損ラベルを補完}します [38-41]。クラス不均衡には重み付き部分クラス (WPC) 損失とロジット調整を使用します [38, 39, 41, 42]。
    %     \item \textbf{第2段階}:\textbf{グローバルモデルを教師}とし、\textbf{整合性正則化}を行うことで、欠損クラス知識の忘却を防ぎます [38-40]。
    % \end{itemize}
    \item \textbf{HAR適用時の課題}:
    \begin{itemize}
        \item 医療データセットで検証されている
        \item \textbf{疑似ラベリングの精度が全体の性能に影響を与える}可能性
    \end{itemize}
\end{itemize}
\end{frame}

% \begin{frame}{マルチラベルFLフレームワーク:FedLGT}
% \begin{itemize}
%     \item 異なるラベル関連性や、陽性ラベルセットの不整合があっても\textbf{堅牢なグローバルモデルを学習}
%     \item \textbf{動作原理}:
%     \begin{itemize}
%         \item \textbf{Universal Label Embedding}: CLIPのような事前学習済みテキストエンコーダで、ラベルの関係係を捉える
%         \item \textbf{Client-Aware Masked Label Embedding}:グローバルモデルからの知識を転移し、ローカルモデルがグローバルモデルの信頼度が低いクラスについてより多く学習
%     \end{itemize}
%     \item \textbf{HAR適用時の課題}:
%     \begin{itemize}
%         \item \textbf{音響・センサーデータへの適用性は未検証}
%         \item 主にマルチラベル画像分類タスクで評価されている
%         \item マルチラベルNon-IID: クラス分布だけでなく、\textbf{ラベルの共起や相関の分布もクライアント間で異なる})という、より複雑なNon-IID問題に直接対応
%     \end{itemize}
% \end{itemize}
% \end{frame}



\begin{frame}{参考文献一覧}
  \small
  \begin{thebibliography}{9}
    \bibitem{ref2} … 
  \end{thebibliography}
\end{frame}
\end{document}