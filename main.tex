\documentclass[unicode,12pt,aspectratio=169,dvipdfmx]{beamer}
\usepackage{bxdpx-beamer}
\usetheme[progressbar=frametitle]{metropolis}
\renewcommand{\kanjifamilydefault}{\gtdefault}
\usepackage{bm}
\usepackage{graphicx}

\title{介護における音響HARと連合学習を用いた異常検知}
\author{竹本志恩}
\institute{INIAD}
\date{\today}
\subject{研究報告 / 輪読}

\AtBeginSection[]{
  \frame{\frametitle{目次}\tableofcontents[currentsection, hideallsubsections]}
}

\begin{document}

%----------------------------------------
% タイトル+目次
%----------------------------------------
\begin{frame}
  \maketitle
  \vspace{0.8cm}
  \tableofcontents
\end{frame}

%----------------------------------------
\section{背景と課題}

\begin{frame}{高齢化と介護現場の課題}
  \begin{itemize}
    \item 世界的な高齢化の加速と介護人材の深刻な不足
    \item 高齢者の転倒や容体急変の早期検知ニーズ
    \item 介護者の負担軽減と入所者の自立支援
  \end{itemize}
\end{frame}

\begin{frame}{プライバシー配慮と非侵襲性の要請}
  \begin{itemize}
    \item カメラベース:高精度だがプライバシー侵害の懸念大
    \item ウェアラブル:装着負担が課題
    \item 環境音(マイク):非接触・安価・プライバシー保護に有利
  \end{itemize}
\end{frame}

%----------------------------------------
\section{FL と HAR の概要}

\begin{frame}{HAR (Human Activity Recognition) とは}
  \begin{itemize}
    \item 人の活動をセンサーデータから自動認識
    \item 応用例:転倒検知、スマートホーム、リハビリ支援 など
    \item センサーモダリティ
    \begin{itemize}
      \item ウェアラブル (加速度計・ジャイロ)
      \item 環境センサー (マイク・PIR・ドアセンサ)
      \item カメラ (高精度だがプライバシー懸念)
    \end{itemize}
  \end{itemize}
\end{frame}

\begin{frame}{Federated Learning (FL) の基本とメリット}
  \begin{itemize}
    \item データを中央サーバに集約せず,各端末で学習→パラメタを共有
    \item メリット
    \begin{itemize}
      \item プライバシー保護:生データはローカルに留まる
      \item パーソナライズ:ユーザ/環境ごとに最適化可能
      \item データ所有権を維持しつつ大規模協調学習
    \end{itemize}
  \end{itemize}
\end{frame}

%----------------------------------------
\section{先行研究サーベイ}

\begin{frame}{FL-HAR が直面する主な技術課題}
  \begin{itemize}
    \item データ非均一性 (Non-IID)
    \item ラベル不足・不均衡
    \item 通信コスト・デバイス異質性
  \end{itemize}
\end{frame}

\begin{frame}{課題と対応技術マトリクス}
  \begin{tabular}{p{0.3\textwidth}p{0.65\textwidth}}
    \hline
    課題 & 主な対応策 \\
    \hline
    Non-IID & FedProx, SCAFFOLD, パーソナライズFL, メタ学習 \\
    ラベル不足 & 合成データ生成, 半教師ありFL, Time-Balanced Focal Loss \\
    通信負荷 & モデル圧縮・量子化, FedDL (動的レイヤー共有) \\
    \hline
  \end{tabular}
\end{frame}

\begin{frame}{セキュリティとマルチラベル異常検知}
  \begin{itemize}
    \item プライバシー攻撃対策:差分プライバシー(DP), 準同型暗号, セキュアアグリゲーション
    \item マルチラベル異常検知の難しさ
    \begin{itemize}
      \item 極端なクラス不均衡
      \item ラベル相関のモデリング
      \item 非IIDデータ下での複雑モデル学習
    \end{itemize}
  \end{itemize}
\end{frame}

%----------------------------------------
\section{提案アプローチ}

\begin{frame}{システム全体フロー}
  \begin{center}
    \includegraphics[width=0.75\linewidth]{example-image} % システム概念図
  \end{center}
  \begin{itemize}
    \item 環境音収集→オンデバイス特徴抽出→FLサーバでモデル集約
  \end{itemize}
\end{frame}

\begin{frame}{モデル設計のポイント}
  \begin{itemize}
    \item Hybrid LSTM–GRU による時間的特徴抽出
    \item マルチラベル分類:Time-Balanced Focal Loss
    \item Non-IID 対策:FedProx, Meta-HAR
  \end{itemize}
\end{frame}

\begin{frame}{プライバシー保護の具体策}
  \begin{itemize}
    \item オンデバイスで生音声を保持し,特徴量のみ転送
    \item 差分プライバシー(DP),セキュアアグリゲーション(SA) の検討
  \end{itemize}
\end{frame}

%----------------------------------------
\section{今後の研究計画}

\begin{frame}{卒業研究段階:基盤技術の確立}
  \begin{itemize}
    \item 異常音マルチラベルモデルの構築・定量評価
    \item データ不均衡対策・Non-IID 対策の比較検証
  \end{itemize}
\end{frame}

\begin{frame}{大学院段階:実用化と拡張}
  \begin{itemize}
    \item IoTデバイス向け軽量化・連合分割学習(FSL)
    \item 時間的文脈を活かした複合異常検知
  \end{itemize}
\end{frame}

%----------------------------------------
\section{まとめ \& 質疑}

\begin{frame}{まとめ}
  \begin{itemize}
    \item 介護現場での音響HAR×FLはプライバシーと実用性を両立
    \item マルチラベル異常検知モデルの独自性
    \item 今後:実環境検証→システム実装へ
  \end{itemize}
\end{frame}

\begin{frame}{ご清聴ありがとうございました/質疑応答}
  \begin{center}
    ご質問をどうぞ
  \end{center}
\end{frame}

\end{document}